\documentclass[headsepline,titlepage,twoside,12pt]{report}
\usepackage[utf8]{inputenc}
\usepackage{amsfonts}
\usepackage[german]{babel}
\usepackage[a4paper,top=4cm,bottom=3cm,left=4cm,right=3cm]{geometry}
\usepackage{bibgerm}
\usepackage{setspace}
\usepackage{paralist}
\usepackage[pdftex]{graphicx}
%\usepackage{floatflt}
\usepackage{graphicx}
\usepackage{setspace}
\usepackage{a4wide}
\usepackage{wrapfig}
\usepackage{latexsym}
\usepackage{amsmath}
\usepackage{subfigure}
\usepackage{amssymb}
\usepackage{mathtools}
\usepackage{tabularx}
\usepackage{textcomp}
\usepackage[headsepline]{scrpage2}
 \usepackage{parskip}
 \usepackage[nooneline]{caption}

\setheadwidth[-10pt]{\textwidth+50pt}
\pagestyle{scrheadings}
\ohead{\pagemark}
\ihead{\textsc{\headmark}}

\let\chaptermarkformat\relax
\let\sectionmarkformat\relax

\automark[section]{chapter}

\usepackage{hyperref}
\hypersetup{colorlinks,%
citecolor=black,%
filecolor=black,%
linkcolor=black,%
urlcolor=black,%
pdftex=black}
%magenta

 \usepackage[margin=10pt,labelfont=bf]{caption}
 \newcommand{\sgn}{\operatorname{sgn}}
  \newcommand{\erf}{\operatorname{erf}}
\newcommand{\bd}{\begin{displaymath}}
\newcommand{\ed}{\end{displaymath}}
\newcommand{\be}{\begin{equation}}
\newcommand{\ee}{\end{equation}}
\newcommand{\intl}{\int \limits}
\newcommand{\suml}{\sum \limits}
\newcommand{\noi}{\noindent}
\renewcommand{\captionfont}{\small}
\newcommand{\x}{{\bar{x}}}
\newcommand{\dg}{^{\dagger}}
\newcommand{\tauh}{\hat{\tau}}
\author{Vorname Name}
\captionsetup{format=hang}

\begin{document}

\bibliographystyle{plain}%Literaturverzeichnis alphabetisch
\onehalfspace

\begin{titlepage}
    \thispagestyle{empty}

\begin{center}

\vspace*{-2cm}

{\Huge UNIVERSIT\"AT LEIPZIG\\[1mm]}
INSTITUT F\"UR MEDIZINISCHE INFORMATIK, STATISTIK UND EPIDEMIOLOGIE\\

\vspace*{1cm}


\vspace{4.0cm}
{\LARGE \textbf{Titel der Arbeit}}\\
\vspace*{3mm}


{\LARGE \textbf{Bachelorarbeit/Masterarbeit}}\\
\vspace*{2mm}
{\Large von Vorname Name}\\
\vspace{4cm}

\parbox{1cm}{
\begin{large}
\begin{tabbing}
Erstgutachter: \hspace{.4cm} \=Prof. Dr. Vorname Name \\[2mm]
Zweitgutachter: \> \=Prof. Dr.Vorname Name\\[2mm]
Abgabedatum: \> 31. September 2013\\
\end{tabbing}
\end{large}}\\
\vspace{5mm}

\end{center}
\end{titlepage}
\newpage
 \thispagestyle{empty}
\qquad
 \newpage

\tableofcontents
\newpage
\chapter{Einleitung}
\section{Gegenstand und Motivation}

Dieses Kapitel hilft nicht nur dem Leser, sondern auch dem Autor zu verstehen, in welchen fachlichen Kontext sich die Arbeit einordnet. Hier muss also insgesamt deutlich werden, wozu die in der Arbeit beschriebenen Arbeitsergebnisse letztlich verwendet werden sollen und wel-chen Nutzen sie möglicherweise bringen können.
Ohne ausreichende Gegenstands- und Motivationsbeschreibung kann ein Leser nicht verstehen, warum z.B. eine entwickelte Software sinnvoll ist bzw. zur Lösung welches Problems sie ver-wendet werden soll. Gerade für die Medizinische Informatik als eine problemorientierte Disziplin ergibt sich der Wert einer Lösung, z.B. einer Software, aber vor allem daraus, ob bzw. wie weit sie ein Problem löst.
Für den Autor bedeutet daher eine unzureichende Gegenstands- und Motivationsbeschreibung die Gefahr, dass er sich die zu lösende Problematik nicht ausreichend klar gemacht hat. Bei der Erstellung der Arbeit besteht dann die Gefahr, dass er möglicherweise methodisch aufregende Lösungen entwirft und realisiert, für die aber ein Problem gar nicht besteht oder die für die Lösung der tatsächlichen Probleme nicht geeignet sind. Trotz einer möglicherweise brillanten Lösung wäre dann doch eine schlechte Bewertung der Lösung und damit der Arbeit zu erwarten.
Aus diesem Grund sollte das Kapitel 1.1 ausführlich sein. Ein Umfang von weniger als drei Seiten wird in der Regel nicht ausreichen.

\subsection{Gegenstand}

\begin{itemize}

\item Welche Situation liegt vor und was soll getan werden?
\item Worum geht es eigentlich?
\item In welcher Welt/Domäne oder welchem Arbeitsbereich/-gebiet bewegen wir uns im Rahmen der Arbeit

\end{itemize}

\subsection{Gegenstand}

\begin{itemize}

\item Warum ist die geschilderte vorliegende Situation problematisch?
\item Worin bestehen die Probleme?
\item Für wen ist sie problematisch?

\end{itemize}


\section{Motivation}

\begin{itemize}

\item Warum lohnt es sich, die genannten Probleme zu lösen?
\item Wer wird welchen Nutzen von dieser Abschlussarbeit haben?
\item Warum ist die Arbeit wichtig?
\item Wer wartet sehnlichst auf die Fertigstellung der Arbeit

\end{itemize}

\section{Problemstellung}

\colorbox{yellow}{Vermeiden Sie Formulierungen wie \glqq Es ist nicht bekannt, ob...\grqq oder \glqq Es existiert kein...\grqq .}

Solche Formulierungen kehren in der Regel einfach das bereits angedachte Lösungsmodell um und postulieren das Fehlen der angedachten Lösung einfach als Problem. Das ist ähnlich, wie wenn es  in der Werbung hieße ?Wenn Sie das Problem haben, dass Ihnen Aspirin fehlt, dann kaufen Sie doch Aspirin?. Sinnvoller ist diese Aussage:  ?Wenn Sie das Problem haben, dass Ihnen der Kopf weh tut, dann kaufen Sie doch Aspirin?. Es ist also bei der Problembeschreibung erforderlich, sich in die Lage dessen zu versetzen, den man mit der angedachten Lösung beglücken möchte. Sein Problem ist zu ermitteln und so zu formulieren, er/sie das Problem wiedererkennt und dadurch geneigt ist, sich für die Lösung des Problems zu interessieren.

\begin{itemize}

\item Welche der in der Problematik geschilderten Probleme sollen im Rahmen dieser Ab-schlussarbeit gelöst werden?
\end{itemize}
Bitte jedes Einzelproblem nummerieren und mit 1-2 Sätzen beschreiben:

\begin{itemize}


\item Problem P1: Problemname 1 mit kurzer Problembeschreibung
\item Problem P2: Problemname 2 mit kurzer Problembeschreibung
\end{itemize}

\section{Zielsetzung}

\begin{itemize}
\item Welche Ergebnisse werden mit dieser Abschlussarbeit angestrebt und welche der o.g. Probleme sollen damit jeweils gelöst werden?
\end{itemize}
Bitte jedes Ziel kurz oder ggf. mit Stichworten beschreiben:
\begin{itemize}
\item Ziel(e)/angestrebte(s) Ergebnis(se) zur Lösung von Problem P1:
	\begin{itemize}
	\item Ziel Z1.1: ....
	\item Ziel Z1.2: ....
	\end{itemize}
\end{itemize}

\section{Aufgabenstellung}

\begin{itemize}
\item Wie sollen die o.g. Ziele erreicht werden?
\item Was soll zur Erreichung der Ziele bzw. zur Schaffung der Ergebnisse getan werden?
\item Welche Fragen müssen zur Erreichung der Ziele bzw. zur Schaffung der Ergebnisse beantwortet  werden?
\end{itemize}


Bitte geben Sie zu jedem der o.g. Ziele mindestens zwei Aufgaben bzw. Fragen an, die bear-beitet bzw. beantwortet werden sollen. Bitte jede Aufgabe bzw. Frage kurz oder ggf. mit Stichworten beschreiben:

\begin{itemize}
\item Aufgaben zu Ziel Z1.1:
	\begin{itemize}
	\item Aufgabe A1.1.1: ....
	\item Aufgabe A1.1.2: ....
	\end{itemize}
\end{itemize}

\section{Aufbau der Arbeit}

\chapter{Grundlagen}

Das Grundlagenkapitel soll den Stand der Forschung erläutern und mit Literatur belegen. Auf diesem Kapitel bauen die Erkenntnisse der Arbeit auf. Gerade in den Grundlagen wird man häufig Quellen benennen, aus denen die Aussagen letztlich stammen. Im Kapitel \glqq Literaturver-zeichnis \grqq dieser Vorlage wird beschrieben wie eine Quellenangabe zu erfolgen hat.
Da sich die Medizinische Informatik mit der Lösung medizinischer Probleme befasst, sollen hier auch die Hintergründe des medizinischen Problems so dargestellt und erläutert werden, dass sie auch für Leser der Arbeit, die nicht Mediziner sind, verständlich sind.
In diesem Kapitel werden auch die Methoden erläutert, die zur Lösung der Probleme eingesetzt wurden. Stellen Sie sicher, dass hier alle, aber auch nur die Grundlagen und Methoden erläutert werden, die in der Arbeit verwendet wurden. Stellen Sie im weiteren Text der Arbeit auch sicher, dass der Leser erkennen kann, wie Sie unter Verwendung der Methoden zu Ihren Er-gebnissen gekommen sind. So sollten z.B. Modellierungsmethoden nur verwendet werden, wenn die Modelle nachvollziehbar dazu genutzt werden, die Ergebnisse zu erzielen.
Bedenken Sie, dass Sie diese Arbeit zum Abschluss eines umfangreichen Studiums schreiben, das vor allem dazu diente, sie mit einem reichen Methodenrepertoire auszustatten. Wählen Sie aus den Methoden, die Sie gelernt haben, aus, benennen Sie die Methoden korrekt und wenden Sie sie an! Aber gehen sie auch kritisch mit dem um, was Ihnen gelehrt wurde. Wenn Sie feststellen, dass gelehrte Methoden ungeeignet sind, diskutieren Sie dies und suchen passendere Methoden! Wenn Sie Methoden benötigen, die nicht gelehrt wurden, suchen Sie nach pas-senden Methoden oder  - wenn Sie nicht fündig werden ? entwickeln Sie die für Ihr Problem passende Methode selbst!


\chapter{Lösungsansatz}

Der Lösungsansatz ist eine kurze Beschreibung der Arbeitshypothese sowie des Vorgehens zur Lösung der in der Einleitung beschriebenen Probleme.

\chapter{Ausführung der Lösung}

Dieses Kapitel kann auch in mehrere Kapitel aufgeteilt werden, wenn das sinnvoll ist!

\section{Unterkapitel n}

\subsection{Unterunterkapitel n}

\chapter{Ergebnisse}

Auch die Ergebnisse können in mehrere Kapitel aufgeteilt oder sogar mit der Präsentation der Ausführung verschmolzen werden. Wichtig ist, dass man zwischen Ausführung und Ergebnis unterscheiden kann.

\chapter{Zusammenfassung}

In der Zusammenfassung wird vor allem zusammenfassend dargestellt, welche Ergebnisse zu den in Kapitel 1.3  formulierten Zielen erreicht wurden. Wurden in Kapitel 1.4  Fragen formuliert, können auch diese hier beantwortet werden. So soll es möglich sein, dass ein Leser von der Arbeit lediglich die Einleitung und die Zusammenfassung lesen und doch die Ergebnisse der Arbeit erfassen kann.

\chapter{Diskussion}

In der Diskussion werden die Ergebnisse der Arbeit kritisch bewertet.
So wäre hier z.B. darzulegen, dass man zwar eine schöne Modellierungsmethode zur Be-schreibung der Kommunikation im Krankenhausinformationssystem gefunden hat, diese aber so aufwändig ist, dass kaum jemand sie benutzen wird, wenn nicht folgende weiteren Arbeiten noch durchgeführt werden.... So wird die Diskussion dann auch einen Ausblick auf die Dinge enthalten, die eigentlich noch zu erledigen sind.
Gerade in einer Seminararbeit könnte hier auch die Kritik des Autors an dem stehen, was er in der Literatur zu dem zu bearbeitenden Thema hier und da gelesen hat.



\begin{thebibliography} {9}
		\bibitem {bmbf},Test, TEST.
\end{thebibliography}

\end{document}


